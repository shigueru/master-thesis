%\usepackage{natbib}
%\usepackage{apalike}
\usepackage{pdfpages}
\usepackage{cite}
\usepackage{tocloft}
\usepackage{subfig}
\renewcommand{\cftfigfont}{Figura }
\renewcommand{\cfttabfont}{Tabla }
\usepackage{mathptm}
\date{\today}
\usepackage{braket}
\usepackage{multirow}
\usepackage{setspace}
\usepackage{verbatim}
\usepackage[a4paper]{geometry}
\geometry{top=3.0cm, bottom=2.5cm, left=3.0cm, right=2.5cm}
\usepackage{graphicx}
\usepackage{amssymb}
\usepackage{float}
\usepackage{lmodern}
\usepackage[spanish,es-lcroman,es-tabla]{babel}
\usepackage[latin1]{inputenc}
\usepackage{appendix}
\usepackage{makeidx}
\usepackage{titlesec}
\usepackage{amsmath}
\usepackage{fancyhdr}
% encabezado paginas normales solo BOOK
% [] es para pares {} es para impares
\lhead[]{} % izquierda
\chead[]{} % centro
\rhead[]{} % derecha
\renewcommand{\headrulewidth}{0pt} % grosor linea de separacion
% pie de pagina paginas normales solo BOOK
\lfoot[]{} % izquierda
\cfoot[\thepage]{\thepage} % centro - pone el numero de la pagina
\rfoot[]{} % derecha
\renewcommand{\footrulewidth}{0pt} % grosor linea de separacion
% encabezado y pie de pagina de los inicio de capitulo
\pagestyle{plain}{
	\fancyhead[L]{} % cabecera lado izquierdo
	\fancyhead[C]{} % cabecera lado centro
	\fancyhead[R]{} % cabecera lado derecha
	\fancyfoot[L]{} % pie pagina lado izquierdo
	\fancyfoot[C]{\thepage} % pie lado centro - pone numero de la pagina
	\fancyfoot[R]{} % pie pagina lado derecho
	\renewcommand{\headrulewidth}{0pt} % cabecera - grosor linea de separacion
	\renewcommand{\footrulewidth}{0pt} % pie pagina - grosor linea de separacion
}
    % opciones del paquete setspace
                         %\doublespacing
                         %\onehalfspace
                         %\singlespace
                         %\spacing{1.5}
                         %===========
\setlength{\parskip}{1cm plus 5mm minus 4mm}  %distancia entre parrafos
\usepackage{color}
\definecolor{gray97}{gray}{.97}
\definecolor{gray75}{gray}{.75}
\definecolor{gray45}{gray}{.45}

\usepackage{listings}
\lstset{ frame=Ltb,
    framerule=0pt,
    aboveskip=0.5cm,
    framextopmargin=3pt,
    framexbottommargin=3pt,
    framexleftmargin=0.4cm,
    framesep=0pt,
    rulesep=.4pt,
    backgroundcolor=\color{gray97},
    rulesepcolor=\color{black},
    %
    stringstyle=\ttfamily\color{purple!40!black},
    showstringspaces = false,
    basicstyle=\small\ttfamily,
    commentstyle=\itshape\color{orange},
    keywordstyle=\bfseries\color{green!40!black},
    %
    numbers=left,
    numbersep=15pt,
    numberstyle=\small,
    numberfirstline = false,
    breaklines=true,
}

\newtheorem{hk}{Teorema}