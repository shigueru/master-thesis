La suite Quantum ESPRESSO utiliza archivos de entrada con formato de texto plano que controlan varios aspectos de la simulaci\'on. A continuaci\'on se muestran los archivos de entrada para cada uno de los pasos realizados en el presente trabajo y partes relevantes de los archivos de salida obtenidos. Se tomara como ejemplo el caso del $YCrO_{3}$.

\noindent El primer paso es realizar un c\'alculo autoconsistente para lo cual el archivo de entrada es el siguiente.

\lstset{breaklines=true}
\begin{lstlisting}
&CONTROL
    calculation = "scf"
    max_seconds =  8.64000e+04
    pseudo_dir  = "/home/ngt/.burai/.pseudopot"
    verbosity = "high"
    restart_mode= "from_scratch"
    outdir = "./"
    prefix = "YCrO3"
/
&SYSTEM
    degauss                   =  0.01
    ecutrho                   =  320.0
    ecutwfc                   =  40.0
    hubbard_u(3)              =  1.13
    hubbard_u(4)              =  1.13
    ibrav                     = 0
    lda_plus_u                = .TRUE.
    nat                       = 20
    nspin                     = 2
    ntyp                      = 4
    nbnd                      = 200
    occupations               = "smearing"
    smearing                  = "gaussian"
    starting_magnetization(1) =  0.00000e+00
    starting_magnetization(2) =  0.00000e+00
    starting_magnetization(3) = -0.8
    starting_magnetization(4) = 0.8
/
&ELECTRONS
    conv_thr         =  1.00000e-06
    diagonalization  = "cg"
    electron_maxstep = 200
    mixing_beta      =  4.00000e-01
    mixing_mode      = "plain"
    startingpot      = "atomic"
    startingwfc      = "atomic+random"
/
K_POINTS {automatic}
6 5 6 0 0 0
CELL_PARAMETERS (angstrom)
    4.996000218   0.000000000  -0.000000233
    0.000000000   5.336857594   0.000000000
   -0.000000393   0.000000000   7.222627972
ATOMIC_SPECIES
Y      88.90585  y_lda_v1.4.uspp.F.UPF
O      15.99940  o_lda_v1.2.uspp.F.UPF
Cr2    51.99610  cr_lda_v1.5.uspp.F.UPF
Cr1    51.99610  cr_lda_v1.5.uspp.F.UPF
ATOMIC_POSITIONS (crystal)
Y        0.020236872   0.926655434   0.750000060
Y        0.520236872   0.573343566   0.250000060
Y        0.479762128   0.426656434   0.749999940
Y        0.979763128   0.073343566   0.249999940
Cr1     -0.000000000   0.500000000   0.500000000
Cr2      0.500000000   0.000000000   0.500000000
Cr1      0.500000000   0.000000000  -0.000000000
Cr2      0.000000000   0.500000000   0.000000000
O        0.688186536   0.303519361   0.448766518
O        0.188186682   0.196480479   0.551233470
O        0.811813464   0.803519361   0.051233482
O        0.311813318   0.696480479   0.948766530
O        0.311813464   0.696480639   0.551233482
O        0.811813318   0.803519521   0.448766530
O        0.188186536   0.196480639   0.948766518
O        0.688186682   0.303519521   0.051233470
O        0.103419514   0.472184554   0.250000004
O        0.603419514   0.027815446   0.750000004
O        0.396580486   0.972183554   0.249999996
O        0.896580486   0.527815446   0.749999996
\end{lstlisting}

\noindent Para iniciar la simulaci\'on se debe ejecutar el programa \textbf{pw.x} y pasarle como par\'ametro el nombre del archivo de entrada y un nombre para el archivo de salida. Como se muestra a continuaci\'on

\begin{lstlisting}
pw.x < YCrO3.scf.in > YCrO3.scf.out
\end{lstlisting}

\noindent El archivo de salida es extenso ya que contiene todas las iteraciones realizadas hasta alcanzar la convergencia. A continuaci\'on se muestra la parte del archivo donde se indica la energ\'ia de fermi y la energ\'ia total del sistema que se obtienen al converger.

\begin{lstlisting}
     the Fermi energy is    15.8431 ev

!    total energy              =   -1462.92649544 Ry
     Harris-Foulkes estimate   =   -1462.92649533 Ry
     estimated scf accuracy    <       0.00000044 Ry

     The total energy is the sum of the following terms:

     one-electron contribution =    -390.25792036 Ry
     hartree contribution      =     309.61691899 Ry
     xc contribution           =    -287.64822073 Ry
     ewald contribution        =   -1094.88316284 Ry
     Hubbard energy            =       0.24588950 Ry
     smearing contrib. (-TS)   =      -0.00000000 Ry

     total magnetization       =     0.00 Bohr mag/cell
     absolute magnetization    =    11.08 Bohr mag/cell

     convergence has been achieved in  10 iterations
\end{lstlisting}

\noindent A continuaci\'on se calcula la densidad de carga para lo cual el archivo de entrada es el siguiente.

\begin{lstlisting}
&inputpp
prefix="YCrO3"
outdir="./"
filplot="YCrO3.rho.dat"
plot_num=0
/
&plot
nfile=1
filepp(1)="YCrO3.rho.dat"
weight(1)=1.0
iflag=3
output_format=3
nx=50
ny=50
nz=50
fileout="YCrO3.plot.rho.xsf"
/
\end{lstlisting}

\noindent El anterior archivo de entrada debe ser pasado como par\'ametro al programa \textbf{pp.x}. El archivo de salida obtenido \textbf{YCrO3.rho.dat} posee un formato que esta dise\~nado para ser le\'ido por \textbf{pp.x} que sirve para realizar un segundo procesamiento y obtener el archivo \textbf{YCrO3.plot.rho.xsf} que posee un formato dise\~nado para ser le\'ido por el programa Xcrysden el cual es usado para graficar la densidad de carga.

\begin{lstlisting}
pp.x < YCrO3.rho.in > YCrO3.rho.out
\end{lstlisting}

\noindent El siguiente paso es calcular la densidad de estados para lo cual el archivo de entrada es el siguiente.

\begin{lstlisting}
&CONTROL
    calculation = "nscf"
    max_seconds =  8.64000e+04
    pseudo_dir  = "/home/ngt/.burai/.pseudopot"
    verbosity = "high"
    restart_mode= "from_scratch"
    outdir = "./"
    prefix = "YCrO3"
/
&SYSTEM
    degauss                   =  0.01
    ecutrho                   =  320.0
    ecutwfc                   =  40.0
    hubbard_u(3)              =  1.13
    hubbard_u(4)              =  1.13
    ibrav                     = 0
    lda_plus_u                = .TRUE.
    nat                       = 20
    nspin                     = 2
    ntyp                      = 4
    nbnd                      = 200
    occupations               = "tetrahedra"
    smearing                  = "gaussian"
    starting_magnetization(1) =  0.00000e+00
    starting_magnetization(2) =  0.00000e+00
    starting_magnetization(3) = -0.8
    starting_magnetization(4) = 0.8
/
&ELECTRONS
    conv_thr         =  1.00000e-06
    diagonalization  = "cg"
    electron_maxstep = 200
    mixing_beta      =  4.00000e-01
    mixing_mode      = "plain"
    startingpot      = "atomic"
    startingwfc      = "atomic+random"
/
K_POINTS {automatic}
8 7 8 0 0 0
CELL_PARAMETERS (angstrom)
    4.996000218   0.000000000  -0.000000233
    0.000000000   5.336857594   0.000000000
   -0.000000393   0.000000000   7.222627972
ATOMIC_SPECIES
Y      88.90585  y_lda_v1.4.uspp.F.UPF
O      15.99940  o_lda_v1.2.uspp.F.UPF
Cr2    51.99610  cr_lda_v1.5.uspp.F.UPF
Cr1    51.99610  cr_lda_v1.5.uspp.F.UPF
ATOMIC_POSITIONS (crystal)
Y        0.020236872   0.926655434   0.750000060
Y        0.520236872   0.573343566   0.250000060
Y        0.479762128   0.426656434   0.749999940
Y        0.979763128   0.073343566   0.249999940
Cr1     -0.000000000   0.500000000   0.500000000
Cr2      0.500000000   0.000000000   0.500000000
Cr1      0.500000000   0.000000000  -0.000000000
Cr2      0.000000000   0.500000000   0.000000000
O        0.688186536   0.303519361   0.448766518
O        0.188186682   0.196480479   0.551233470
O        0.811813464   0.803519361   0.051233482
O        0.311813318   0.696480479   0.948766530
O        0.311813464   0.696480639   0.551233482
O        0.811813318   0.803519521   0.448766530
O        0.188186536   0.196480639   0.948766518
O        0.688186682   0.303519521   0.051233470
O        0.103419514   0.472184554   0.250000004
O        0.603419514   0.027815446   0.750000004
O        0.396580486   0.972183554   0.249999996
O        0.896580486   0.527815446   0.749999996
\end{lstlisting}

\noindent Este archivo debe ser pasado como par\'ametro del programa \textbf{pw.x} e indicando un archivo de salida.

\begin{lstlisting}
pw.x < YCrO3.nscf.in > YCrO3.nscf.out
\end{lstlisting}

\noindent A continuaci\'on para poder extraer la informaci\'on correspondiente a la densidad de estados total y parcial de cada elemento se debe usar el programa \textbf{projwfc.x} con el siguiente archivo de entrada.

\begin{lstlisting}
&PROJWFC
prefix="YCrO3"
outdir="./"
ngauss=0
degauss=0.01
Emin=-40
Emax=32
DeltaE=0.1
filpdos="YCrO3.dos-proyec"
filproj="YCrO3.proyeccion.dos"
/
\end{lstlisting}

\noindent Luego ejecutamos \textbf{projwfc.x}

\begin{lstlisting}
projwfc.x < YCrO3.dos.proyeccion.in > YCrO3.dos.proyeccion.out
\end{lstlisting}

\noindent Los archivos resultantes de este paso son luego procesados por el programa escrito en python suma\_pdos.py para realizar las gr\'aficas de densidad de estados. Se cre\'o un script de shell que automatiza la ejecuci\'on del programa suma\_pdos.py, este script se encuentra en el ap\'endice de programas auxiliares.

\noindent El siguiente paso es calcular la estructura de bandas para lo cual el archivo de entrada es el siguiente.

\begin{lstlisting}
&CONTROL
    calculation = "bands"
    max_seconds =  8.64000e+04
    pseudo_dir  = "/home/ngt/.burai/.pseudopot"
    verbosity = "high"
    restart_mode= "from_scratch"
    outdir = "./"
    prefix = "YCrO3"
/
&SYSTEM
    ibrav                     = 8
    a                         =  4.996000218
    b                         =  5.336857594
    c                         =  7.222627972
    degauss                   =  0.01
    ecutrho                   =  320.0
    ecutwfc                   =  40.0
    hubbard_u(3)              =  1.13
    hubbard_u(4)              =  1.13
    lda_plus_u                = .TRUE.
    nat                       = 20
    nspin                     = 2
    ntyp                      = 4
    nbnd                      = 200
    occupations               = "smearing"
    smearing                  = "gaussian"
    starting_magnetization(1) =  0.00000e+00
    starting_magnetization(2) =  0.00000e+00
    starting_magnetization(3) = -0.8
    starting_magnetization(4) = 0.8
/
&ELECTRONS
    conv_thr         =  1.00000e-06
    diagonalization  = "cg"
    electron_maxstep = 200
    mixing_beta      =  4.00000e-01
    mixing_mode      = "plain"
    startingpot      = "atomic"
    startingwfc      = "atomic+random"
/
K_POINTS {tpiba_b}
5
gG     20
X      20
S      20
Y      20
gG     0
ATOMIC_SPECIES
Y      88.90585  y_lda_v1.4.uspp.F.UPF
O      15.99940  o_lda_v1.2.uspp.F.UPF
Cr2    51.99610  cr_lda_v1.5.uspp.F.UPF
Cr1    51.99610  cr_lda_v1.5.uspp.F.UPF
ATOMIC_POSITIONS (crystal)
Y        0.020236872   0.926655434   0.750000060
Y        0.520236872   0.573343566   0.250000060
Y        0.479762128   0.426656434   0.749999940
Y        0.979763128   0.073343566   0.249999940
Cr1     -0.000000000   0.500000000   0.500000000
Cr2      0.500000000   0.000000000   0.500000000
Cr1      0.500000000   0.000000000  -0.000000000
Cr2      0.000000000   0.500000000   0.000000000
O        0.688186536   0.303519361   0.448766518
O        0.188186682   0.196480479   0.551233470
O        0.811813464   0.803519361   0.051233482
O        0.311813318   0.696480479   0.948766530
O        0.311813464   0.696480639   0.551233482
O        0.811813318   0.803519521   0.448766530
O        0.188186536   0.196480639   0.948766518
O        0.688186682   0.303519521   0.051233470
O        0.103419514   0.472184554   0.250000004
O        0.603419514   0.027815446   0.750000004
O        0.396580486   0.972183554   0.249999996
O        0.896580486   0.527815446   0.749999996
\end{lstlisting}

\noindent El archivo anterior debe ser pasado como par\'ametro del programa \textbf{pw.x} del siguiente modo.

\begin{lstlisting}
pw.x < YCrO3.bandas.in > YCrO3.bandas.out
\end{lstlisting}

\noindent A continuaci\'on para extraer la informaci\'on de las bandas se utiliza el programa \textbf{bands.x} con el siguiente archivo de entrada.

\begin{lstlisting}
&BANDS
prefix="YCrO3"
outdir="./"
filband="YCrO3.bandas.general"
/
\end{lstlisting}

\noindent Luego ejecutamos \textbf{bands.x}.

\begin{lstlisting}
bands.x < YCrO3.ext-band-gen.in > YCrO3.ext-band-gen.out
\end{lstlisting}

\noindent El paso anterior es el \'ultimo en el proceso de simulaci\'on.