Para el $BiFeO_{3}$ se observ\'o un gap de energ\'ia de $1.4$ eV y $1.8$ eV 
para los arreglos antiferromagn\'eticos tipo A y G respectivamente. Adem\'as  
el comportamiento de ambos arreglos 
antiferromagn\'eticos es similar, esto se puede atribuir a que los orbitales 
\textbf{d} del hierro 
son los que poseen la mayor densidad cerca del nivel de fermi en la banda de 
conducci\'on y en la banda de valencia cerca del nivel de fermi son los 
orbitales \textbf{p} del ox\'igeno los que poseen la mayor densidad. Adem\'as 
en ambos arreglos antiferromagn\'eticos se observa que los estados de los tres 
elementos se encuentran mezclados a lo largo de todo el rango de energ\'ias. De 
los resultados obtenidos se concluy\'o que el arreglo antiferromagn\'etico tipo 
C para el $BiFeO_{3}$ no converge.


\noindent En el caso del $YCrO_{3}$ se observ\'o un gap de energ\'ia de $1.3$ 
eV, $1.32$ eV y $1.6$ eV 
para los arreglos antiferromagn\'eticos tipo A, C y G respectivamente. Adem\'as 
los tres arreglos antiferromagn\'eticos presentan un comportamiento similar, 
esto se puede atribuir a que 
los orbitales \textbf{d} del cromo son los que poseen la mayor densidad de 
estados cerca del nivel de fermi tanto en la banda de valencia como en la banda 
de conducci\'on. Tambi\'en se observa que los estados de los tres elementos se 
encuentran mezclados a lo largo de todo el rango de energ\'ias.
En el caso de arreglo antiferromagn\'etico tipo G del $YCrO_{3}$ se observ\'o 
que presenta un segundo gap por sobre el nivel de fermi, una caracter\'istica 
que no se encuentra presente en los otros dos tipos de arreglos 
antiferromagn\'eticos.

\noindent Los resultados obtenidos pueden considerarse como punto de partida 
para futuras investigaciones en las cuales se pueden introducir nuevos 
par\'ametros como la temperatura, es decir simular fases tanto del $BiFeO_{3}$ 
como del $YCrO_{3}$ que se presenten a temperaturas m\'as altas. Adem\'as es 
posible estudiar la variaci\'on de la magnetizaci\'on en ambos materiales 
haciendo uso del m\'etodo monte carlo.