Se estudia la estructura electr\'onica de las perovskitas $BiFeO_{3}$ e
$YCrO_{3}$. Se utiliz\'o una disposici\'on antiferromagn\'etica para los 
\'atomos de 
ambos materiales. Para el $BiFeO_{3}$ se utilizaron los arreglos 
antiferromagn\'eticos tipo A y G. Para el $YCrO_{3}$ se utilizaron los arreglos 
antiferromagn\'eticos tipo A, C y G. El trabajo 
se desarrollo en el marco de la teor\'ia del funcional de densidad, para lo 
cual se utiliz\'o el paquete de simulaci\'on Quantum Espresso y 
pseudopotenciales ultrasuaves junto a la aproximaci\'on de densidad local 
considerando adem\'as al par\'ametro de Hubbard con un valor de $2.43$ eV y 
$1.13$ 
eV para los \'atomos de hierro y cromo respectivamente. Para el 
$BiFeO_{3}$ se obtuvo un gap de energ\'ia de $1.4$ eV para el arreglo tipo A y 
un 
gap de $1.8$ eV para el arreglo tipo G. Para el $YCrO_{3}$ se obtuvo un gap de 
energ\'ia de $1.30$ eV para el arreglo tipo A, para el arreglo tipo C se obtuvo 
un gap de $1.32$ eV y para el arreglo tipo G se obtuvo un gap de $1.6$ eV. Se 
determin\'o que el arreglo antiferromagn\'etico tipo G es el m\'as estable para 
ambos materiales.
