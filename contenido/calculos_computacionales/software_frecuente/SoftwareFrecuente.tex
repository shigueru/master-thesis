Existen muchos paquetes de simulaci\'on que implementan la teor\'ia del 
funcional de densidad. Estos paquetes de simulaci\'on se diferencian por el 
conjunto de funciones base que utilizan, los tipos de pseupotenciales que 
admiten y las propiedades que pueden calcular, entre otros aspectos. A 
continuaci\'on se presenta una lista de algunos paquetes de simulaci\'on 
relevantes.

\begin{itemize}
    \item \textbf{VASP:}  Es un paquete de simulaci\'on para modelamiento de 
    materiales a escala at\'omica como c\'alculos de estructura electr\'onica y 
    din\'amica molecular mec\'anico-cu\'antica a partir de primeros principios. 
    Implementa la teor\'ia del funcional de densidad, funcionales h\'ibridos, 
    teor\'ia de perturbaci\'on y el m\'etodo de funciones de Green. Adem\'as 
    utiliza pseudopotenciales, m\'etodo del proyector de ondas aumentadas y 
    ondas planas.
    \item \textbf{WIEN2k:} Es un paquete de simulaci\'on que permite realizar 
    c\'alculos de estructura electr\'onica de s\'olidos usando la teor\'ia del 
    funcional de densidad. Implementa el m\'etodo de ondas planas aumentadas 
    linealizadas m\'as orbitales locales (LAPW + lo) y tambi\'en es capaz de 
    considerar efectos relativistas.
    \item \textbf{ABINIT:} Es un paquete de simulaci\'on para c\'alculos de 
    estructura electr\'onica que implementa la teor\'ia del funcional de 
    densidad con pseudopotenciales o wavelets. Adem\'as implementa la teor\'ia 
    de perturbaciones y la teor\'ia del funcional de densidad dependiente del 
    tiempo.
    \item \textbf{Quantum ESPRESSO:} Es una suite de paquetes de simulaci\'on 
    para c\'alculos de estructura electr\'onica que implementa la teor\'ia del 
    funcional de densidad utilizando pseudopotenciales y ondas planas. Adem\'as 
    implementa la teor\'ia del funcional de densidad dependiente del tiempo. 
    Este es el software que se utiliza en el presente estudio.
\end{itemize}