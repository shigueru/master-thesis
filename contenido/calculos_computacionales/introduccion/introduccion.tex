La soluci\'on de la ecuaci\'on de Kohn-Sham implica un c\'alculo de muchas 
variables por lo que se usa el m\'etodo autoconsistente. En la soluci\'on de esta ecuaci\'on se usa la expansi\'on en ondas planas y se aproxima el potencial at\'omico con un pseudopotencial e involucra el c\'alculo de integrales en el espacio rec\'iproco. A continuaci\'on se muestra el criterio seguido para cortar el desarrollo de ondas planas y se discuten algunos aspectos relacionados con los m\'etodos computacionales.