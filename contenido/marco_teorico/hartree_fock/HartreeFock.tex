En el m\'etodo de Hartree-Fock la funci\'on de onda es aproximada como una 
combinaci\'on lineal de funciones de onda mono-electr\'onicas, en la forma de 
un determinante de Slater

\begin{equation}
\psi (r_{1}r_{2}\cdots r_{N}) = \frac{1}{\sqrt{N!}} 
\left ( 
\begin{array}{cccc}
\phi _{1}(r_{1}) & \phi _{2}(r_{1}) & \cdots & \phi _{N}(r_{1}) \\
\phi _{1}(r_{2}) & \phi _{2}(r_{2}) & \cdots & \phi _{N}(r_{2}) \\
\vdots & \vdots & \ddots & \vdots \\
\phi _{1}(r_{N}) & \phi _{2}(r_{N}) & \cdots & \phi _{N}(r_{N})
\end{array} 
\right ) \textrm{ ,}
\end{equation}

\noindent donde $\frac{1}{\sqrt{N!}}$ es el factor de normalizaci\'on para un 
sistema de 
N electrones.

\noindent Usando el determinante de Slater, podemos escribir la ecuaci\'on de 
onda del 
siguiente modo

%\begin{equation}
%\big [ - \sum _{i}^{n} \frac{1}{2} \nabla _{i}^{2} - \sum _{i} ^{N} 
%\frac{Z_{i}}{|R_{i}-r|} + \sum _{i \ne j} ^{n} \int \frac{|\phi 
%    _{i}(r^{\prime})|^{2}}{|r-r^{\prime }|}d^{3}r^{\prime} \big] \phi _{j} (r) 
%    - 
%\sum _{i} \int dr^{\prime} \frac{\phi _{i}^{*}(r^{\prime}) \phi 
%    _{j}(r^{\prime})}{|r^{\prime}-r|} \phi_{i}(r) = E_{j} \phi _{j}(r) 
%%%\textrm{ 
%    ,}
%\end{equation}

\begin{equation}
\big [ - \sum _{i}^{n} \frac{1}{2} \nabla _{i}^{2} - \sum _{i} ^{N} 
\frac{Z_{i}}{|R_{i}-r|} + \sum _{i \ne j} ^{n} \int \frac{|\phi 
    _{i}(r^{\prime})|^{2}}{|r-r^{\prime }|}d^{3}r^{\prime} \big] \phi _{j} (r) 
- 
\sum _{i} \int dr^{\prime} \frac{\phi _{i}^{*}(r^{\prime}) \phi 
    _{j}(r^{\prime})}{|r^{\prime}-r|} \phi_{i}(r) = E_{j} \phi _{j}(r) \textrm{ 
    ,}
\end{equation}

\noindent donde el \'ultimo t\'ermino del lado izquierdo corresponde al 
intercambio, el 
cual es un tipo de interacci\'on cu\'antica y no existe un an\'alogo cl\'asico 
de este t\'ermino.
