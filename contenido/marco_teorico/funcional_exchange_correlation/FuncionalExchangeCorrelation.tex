En la secci\'on anterior se asumi\'o que la energ\'ia de 
intercambio-correlaci\'on  
pod\'ia expresarse como un funcional de la densidad electr\'onica. La energ\'ia 
de intercambio-correlaci\'on representa un aproximado del diez por ciento de la 
energ\'ia total, pero es fundamental para calcular propiedades de los 
materiales,tales como: enlaces, polarizaci\'on del spin y el gap de las bandas 
de energ\'ia; entre otras. 

\noindent La energ\'ia de intercambio-correlaci\'on recoge todas las 
interacciones de un 
electr\'on con los dem\'as, que se encuentran en el sistema real de electrones 
interactuantes, por lo que se debe aproximar esta energ\'ia lo mejor 
posible.

%             =
%            = =
%           =   =
%          =     =
%         =       = 
% ========         ==========
%    HUECO DE INTERCAMBIO
% ========         ===========
%         =       =
%          =     =
%           =   =
%            = =
%             =

\subsection{Hueco de intercambio}

Este concepto surge de la propiedad de antisimetr\'ia de los orbitales, la cual 
indica que dos electrones con el mismo espin no pueden ocupar el mismo orbital. 
Esta restricci\'on genera una separaci\'on espacial entre los electrones 
provocando una menor repulsi\'on entre ellos a la vez que causa una reducci\'on 
en la densidad electr\'onica. A esta densidad electr\'onica reducida se le 
denomina hueco de intercambio, se puede ver una representaci\'on de este en la 
figura \ref{HuecoIntercambio}. La energ\'ia de intercambio representa la 
interacci\'on entre un hueco de intercambio y la densidad electr\'onica a lo 
largo de cierta distancia.

\begin{figure}[H]
    \centering
    \includegraphics[width=0.4\textwidth]{contenido/marco_teorico/funcional_exchange_correlation/img_exchange_correlation/HuecoIntercambio.png}
    \caption[Hueco de intercambio]{Representaci\'on del hueco de intercambio 
    entre dos electrones.}
    \label{HuecoIntercambio}
\end{figure}

%             =
%            = =
%           =   =
%          =     =
%         =       = 
% ========         ==========
%    HUECO DE CORRELACION
% ========         ===========
%         =       =
%          =     =
%           =   =
%            = =
%             =

\subsection{Hueco de correlaci\'on}

Se sabe que dos electrones con diferente espin pueden ocupar el mismo orbital, 
pero se repelen entre si debido a que poseen la misma carga. Esto se puede 
entender como una correlaci\'on electr\'onica que reduce la densidad 
electr\'onica alrededor del electr\'on, generando una energ\'ia de atracci\'on 
peque\~na. 
Este efecto se conoce como hueco de correlaci\'on. Se puede ver una 
representaci\'on de este efecto en la figura \ref{HuecoCorrelacion}. El hueco 
de correlaci\'on tiene direcciones negativas y positivas porque proviene de la 
interacci\'on de dos electrones con espines opuestos.

\begin{figure}[H]
    \centering
    \includegraphics[width=0.4\textwidth]{contenido/marco_teorico/funcional_exchange_correlation/img_exchange_correlation/HuecoCorrelacion.png}
    \caption[Hueco de correlaci\'on]{Representaci\'on del hueco de 
    correlaci\'on entre dos electrones.}
    \label{HuecoCorrelacion}
\end{figure}

%             =
%            = =
%           =   =
%          =     =
%         =       = 
% ========         ==========
% HUECO DE INTERCAMBIO-CORRELACION
% ========         ===========
%         =       =
%          =     =
%           =   =
%            = =
%             =

\subsection{Hueco de intercambio-correlaci\'on}

El hueco de inetrcambio-correlaci\'on surge de la uni\'on de los huecos de 
intercambio y correlaci\'on. En presencia de densidades altas de electrones, se 
puede considerar que la parte de intercambio contribuye m\'as, ya que esta 
parte proviene del principio de exclusi\'on de Pauli que predomina cuando los 
electrones est\'an m\'as cerca entre si. En cambio, con densidades bajas de 
electrones, la parte de la correlaci\'on llega a ser comparable a la parte de 
intercambio.


\noindent Se considera que la energ\'ia de intercambio-correlaci\'on es una 
funcional de 
la densidad electr\'onica de car\'acter local o semilocal, luego la energ\'ia 
de intercambio-correlaci\'on por electr\'on podemos definirla como la energ\'ia 
de interacci\'on electrost\'atica entre un electr\'on en $r$ con la densidad  
del hueco de intercambio-correlaci\'on en $r^{\prime }$.

\begin{equation}
    \varepsilon _{xc} [\rho (r)] = \frac{1}{2} \int \frac{\rho 
    ^{hueco}_{xc}(r,r^{\prime })}{|r - r^{\prime }|} dr^{\prime }
\end{equation}

\noindent Entonces la energ\'ia de intercambio-correlaci\'on total es la 
integral sobre todo el espacio del producto de la densidad electr\'onica con la 
energ\'ia de intercambio-correlaci\'on por electr\'on.

\begin{equation}
    E_{xc} [\rho (r)] = \int \rho (r) \varepsilon_{xc} dr
\end{equation}

Reemplazando $\varepsilon_{xc}$ obtenemos

\begin{equation}
    E_{xc} [\rho (r)] = \frac{1}{2} \int \int \frac{\rho (r) \rho 
        ^{hueco}_{xc}(r,r^{\prime })}{|r - r^{\prime }|} dr dr^{\prime }
\end{equation}

\noindent Al considerar que la energ\'ia de intercambio-correlaci\'on es de 
car\'acter local o semilocal durante el proceso de aproximaci\'on, se facilitan 
muchos los c\'alculos. A lo largo del tiempo se han propuesto muchos tipos de 
aproximaciones para la energ\'ia de intercambio-correlaci\'on, de los cuales se 
pueden separar dos que son los m\'as usados en distintos tipos de aplicaciones, 
los cuales son: Aproximaci\'on de densidad local (LDA) y aproximaci\'on de 
gradiente generalizado (GGA). En la siguiente secci\'on se detalla la 
aproximaci\'on LDA.