Una descripci\'on precisa de las propiedades f\'isicas y qu\'imicas de los 
materiales requiere del tratamiento mec\'anico-cu\'antico de un sistema de 
muchas part\'iculas, formado por electrones y n\'ucleos. Adem\'as se debe 
solucionar la ecuaci\'on de Schrodinger con el hamiltoniano mostrado en ( 
\ref{schrodinger}) para una funci\'on de onda con una cantidad de 
variables espaciales $\{R,r\}$ equivalente a tres veces el n\'umero total de 
part\'iculas del sistema.

\begin{equation} \label{schrodinger}
-\sum _{I=1}^{N} \frac{1}{2} \nabla _{I}^{2} - \sum 
_{i=1}^{n} \frac{1}{2} \nabla _{i}^{2} + \frac{1}{2} \sum _{I \ne 
J}^{N} \frac{Z_{I}Z_{J}}{|R_{I}-R_{J}|} + \frac{1}{2} \sum _{i\ne j}^{n} 
\frac{1}{|r_{i}-r_{j}|} - \sum _{I}^{N} \sum _{j}^{n} 
\frac{Z_{I}}{|R_{I}-r_{j}|}
\end{equation}

Donde:

\begin{itemize}
    \item $\displaystyle -\sum _{I=1}^{N} \frac{1}{2} \nabla _{I}^{2}$ : 
    Energ\'ia cin\'etica de los n\'ucleos.
    \item $\displaystyle - \sum _{i=1}^{n} \frac{1}{2} \nabla _{i}^{2}$ : 
    Energ\'ia cin\'etica de los electrones.
    \item $\displaystyle \frac{1}{2} \sum _{I \ne J}^{N} 
    \frac{Z_{I}Z_{J}}{|R_{I}-R_{J}|}$ : 
    Interacci\'on n\'ucleo-n\'ucleo.
    \item $\displaystyle \frac{1}{2} \sum _{i\ne j}^{n} 
    \frac{1}{|r_{i}-r_{j}|}$ : 
    interacci\'on electr\'on-electr\'on.
    \item $\displaystyle \sum _{I}^{N} \sum _{j}^{n} 
    \frac{Z_{I}}{|R_{I}-r_{j}|}$ : 
    Interacci\'on n\'ucleo-electr\'on.
\end{itemize}

\noindent El gran n\'umero de variables contenidas en el hamiltoniano 
(\ref{schrodinger}) 
hace dif\'icil solucionar la ecuaci\'on de Schrodinger para obtener 
informaci\'on del sistema. Entonces se debe usar una aproximaci\'on.

\noindent La aproximaci\'on de Born-Oppenheimer considera la diferencia de 
masas entre 
los n\'ucleos y los electrones, la cual es grande, por lo que se puede 
considerar que los electrones responden de manera inmediata al movimiento de 
los n\'ucleos. Basado en lo anterior, los n\'ucleos pueden ser tratados como 
part\'iculas fijas en el espacio, que son fuente de un potencial externo en el 
cual los electrones se mueven. De esta manera podemos expresar el hamiltoniano 
de Born-Oppenheimer utilizando unidades at\'omicas como

\begin{equation}
H_{BO} =-\sum _{i=1}^{N} \frac{1}{2} \nabla _{i}^{2} + \frac{1}{2} \sum _{I \ne 
    J}^{N} \frac{Z_{I}Z_{J}}{|R_{I}-R_{J}|} + \frac{1}{2} \sum _{i\ne j}^{n} 
\frac{1}{|r_{i}-r_{j}|} - \sum _{I}^{N} \sum _{j}^{n} 
\frac{Z_{I}}{|R_{I}-r_{j}|}
\label{born-oppenheimer}
\end{equation}
