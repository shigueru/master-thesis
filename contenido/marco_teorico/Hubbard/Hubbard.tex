La aproximaci\'on de densidad local falla por si misma al describir sistemas 
que poseen electrones fuertemente correlacionados. En estos sistemas, por 
ejemplo materiales que posean \'atomos de metales de transici\'on o tierras 
raras, los electrones de los orbitales \textbf{d} y \textbf{f} se encuentran 
fuertemente localizados, y su comportamiento se desv\'ia del descrito por el 
modelo de gas de electrones homog\'eneo. Estos electrones localizados perciben 
una fuerte interacci\'on de coulomb que no es descrita por la aproximaci\'on de 
densidad local. Por lo que con el fin de considerar esta fuerte interacci\'on 
de 
coulomb,es necesaria una modificaci\'on del funcional.

\noindent La aproximaci\'on de densidad local corregida para tratar con los 
electrones 
fuertemente correlacionados es llamada LDA+U. La idea detr\'as de LDA+U es 
incluir en el funcional un termino que tome en cuenta las interacciones 
electron-electron fuertes en los orbitales localizados. La energ\'ia del 
sistema con esta correcci\'on seria la siguiente.

\begin{equation}\label{hubbard_correcion}
E^{LDA+U} [\rho (r)]= E^{LDA} [\rho (r)] + 
E^{Hubbard}[n_{mm^{\prime}}^{I\sigma}] - E^{dc}[n^{I\sigma}] \textrm{ ,}
\end{equation}

\noindent donde $n_{mm^{\prime}}^{I\sigma}$ es la matriz de ocupaci\'on de los 
orbitales 
at\'omicos de los \'atomos con electrones fuertemente correlacionados en el 
sitio I con espin $\sigma$.

\begin{equation}
n_{mm^{\prime}}^{I\sigma} = \sum _{k,n} f_{k,n} \braket{\psi _{k,n}^{\sigma } | 
\phi _{m}^{I}} \braket{\phi _{m^{\prime}}^{I} | \psi _{k,n}^{\sigma }} \textrm{ 
,}
\end{equation}

\noindent Donde $f_{k,n}$ es la ocupaci\'on de los estados electr\'onicos con 
vector de 
onda \textbf{k} e \'indice de banda n. El $\phi _{m}^{I}$ es el m-\'esimo 
orbital 
at\'omico en el sitio I, y $\psi _{k,n}^{\sigma }$ es la funci\'on  de onda 
electr\'onica correspondiente al estado (k,n) con espin $\sigma $.



\noindent El primer t\'ermino de la ecuaci\'on \ref{hubbard_correcion} es el 
funcional de 
energ\'ia en la aproximaci\'on de densidad local. El segundo t\'ermino de la 
ecuaci\'on \ref{hubbard_correcion} $E^{Hubbard}[n_{mm^{\prime}}^{I\sigma}]$ es 
el t\'ermino que cuantifica la correlaci\'on de los estados ocupados de los 
orbitales. El tercer t\'ermino de la ecuaci\'on \ref{hubbard_correcion} 
$E^{dc}[n^{I\sigma}]$ es la energ\'ia de correlaci\'on, la cual es restada de 
la energ\'ia total para evitar un doble conteo. Y $n^{I\sigma} = \sum _{m} 
n_{mm}^{I\sigma}$ es la traza de la matriz de ocupaciones de los orbitales 
at\'omicos con electrones fuertemente correlacionados. Se puede reescribir la 
ecuaci\'on \ref{hubbard_correcion} como

\begin{equation}\label{hubb_rescrito}
E^{LDA+U} [\rho (r)]= E^{LDA} [\rho (r)] + \sum _{I} \left[ \frac{U}{2} \sum 
_{m,\sigma \neq m^{\prime},\sigma ^{\prime}} n_{m}^{I\sigma} 
n_{m^{\prime}}^{I\sigma ^{\prime}} - \frac{U}{2} n^{I}(n^{I}-1) \right] 
\textrm{ ,}
\end{equation}

\noindent donde $n_{m}^{I\sigma} = n_{mm^{\prime}}^{I\sigma}$ y $n^{I} = \sum 
_{m,\sigma 
} n_{m}^{I\sigma }$. \textbf{U} es el par\'ametro de Hubbard que toma en 
cuenta las correlaciones entre los electrones fuertemente localizados.