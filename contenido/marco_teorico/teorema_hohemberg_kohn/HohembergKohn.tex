Hohenberg y Kohn  en 1964 finalmente probaron que la densidad electr\'onica 
tiene un papel fundamental en los c\'alculos de estructura electr\'onica. 
Enunciaron dos teoremas que completaron la relaci\'on entre la densidad 
electr\'onica, energ\'ia externa, hamiltoniano y funci\'on de onda.

%             =
%            = =
%           =   =
%          =     =
%         =       = 
% ========         ==========
%        TEOREMA UNO
% ========         ===========
%         =       =
%          =     =
%           =   =
%            = =
%             =

\begin{hk}
    El potencial $V_{ext}(r)$ est\'a determinado \'unicamente, excepto por una 
    constante, por la densidad del estado fundamental $\rho _{0}(r)$.
\end{hk}

\noindent Para poder comprobarlo, se asume que existen dos potenciales externos 
$V_{ext}^{(1)}(r)$ y $V_{ext}^{(2)}(r)$ diferentes que generan la misma 
densidad en el estado fundamental $\rho _{0}(r)$. 
Adem\'as los dos potenciales definen dos hamiltonianos $H^{(1)}$ y $H^{(2)}$ 
para los cuales se definen dos funciones de onda diferentes en el estado 
fundamental $\Psi ^{(1)}$ y $\Psi ^{(2)}$ respectivamente.

\noindent Luego, como $\Psi ^{(2)}$ no pertenece al estado fundamental de 
$H^{(1)}$, 
tenemos lo siguiente
\begin{equation} \label{comparacion_uno_dos}
\bra{\Psi ^{(1)}} H^{(1)} \ket{\Psi ^{(1)}} < \bra{\Psi ^{(2)}} H^{(1)} 
\ket{\Psi ^{(2)}} \textrm{ ,}
\end{equation}
\noindent donde
\begin{equation}
\bra{\Psi ^{(1)}} H^{(1)} \ket{\Psi ^{(1)}} = E^{(1)}
\end{equation}
y
\begin{equation}
\begin{split}
\bra{\Psi ^{(2)}} H^{(1)} \ket{\Psi ^{(2)}} &= \bra{\Psi ^{(2)}} H^{(2)} 
\ket{\Psi ^{(2)}} + \bra{\Psi ^{(2)}} H^{(1)} - H^{(2)} \ket{\Psi ^{(2)}} \\
&= E^{(2)} + \int d^{3}r [ V_{ext}^{(1)}(r) - V_{ext}^{(2)}(r) ] \rho 
_{0}(r)
\end{split} \textrm{ ,}
\end{equation}
\noindent entonces de \ref{comparacion_uno_dos}
\begin{equation}
E^{(1)} < E^{(2)} + \int d^{3}r [ V_{ext}^{(1)}(r) - V_{ext}^{(2)}(r) ] 
\rho _{0}(r) \label{ec1}
\end{equation}
\noindent De igual forma como $\Psi ^{(1)}$ no pertenece al estado fundamental 
de 
$H^{(2)}$, tenemos
\[
\bra{\Psi ^{(2)}} H^{(2)} \ket{\Psi ^{(2)}} < \bra{\Psi ^{(1)}} H^{(2)} 
\ket{\Psi ^{(1)}}
\]
\noindent Entonces
\begin{equation}
E^{(2)} < E^{(1)} - \int d^{3}r [ V_{ext}^{(1)}(r) - V_{ext}^{(2)}(r) ] 
\rho _{0}(r) \label{ec2}
\end{equation} 
\noindent Sumando las expresiones \ref{ec1} y \ref{ec2}, obtenemos
\begin{equation}
E^{(1)} + E^{(2)} < E^{(2)} + E^{(1)}
\end{equation}
\noindent Lo cual es una contradicci\'on, por lo que concluimos que no es 
posible que 
existan dos potenciales externos diferentes que 
se relacionen con una misma densidad en el estado fundamental.

%             =
%            = =
%           =   =
%          =     =
%         =       = 
% ========         ==========
%        TEOREMA DOS
% ========         ===========
%         =       =
%          =     =
%           =   =
%            = =
%             =

\begin{hk}
    La densidad que minimiza la energ\'ia total es exactamente la densidad del 
    estado fundamental.
\end{hk}

\noindent Para Hohenberg-khon la energ\'ia se puede expresar como:

\begin{equation}
\begin{split}
E_{HK}[\rho] &= T[\rho] + E_{int}[\rho] + \int d^{3}rV_{ext}(r)\rho (r) + 
E_{II} \\
&= F_{HK}[\rho] + \int d^{3}r V_{ext}(r)\rho (r) + E_{II} \textrm{ ,}
\end{split}
\end{equation}

\noindent donde:
\[
\begin{split}
T[\rho] & \text{  : es el funcional de energ\'ia cin\'etica} \\
E_{II}  & \text{  : es el funcional de energ\'ia de interacci\'on entre 
    n\'ucleos} \\
E_{int}[\rho] & \text{  : es la energ\'ia potencial del sistema de 
    electrones que interactuan}
\end{split}
\]
\noindent Adem\'as
\begin{equation}
F_{HK}[\rho] = T[\rho] + E_{int}[\rho] \label{F_HK}
\end{equation}
\noindent Es importante notar que $F_{HK}[\rho]$ es una funcional universal, es 
decir 
que es la misma para todos los sistemas electr\'onicos, siendo independiente
del potencial externo, ya que la energ\'ia cin\'etica y la energ\'ia de
interacci\'on $E_{int}[\rho]$ solo dependen de la densidad.


\noindent Considerando un sistema con una densidad electr\'onica en el estado 
fundamental
$\rho_{0}^{(1)}$ que corresponde a un potencial externo $V_{ext}^{(1)}(r)$,
la energ\'ia del estado fundamental es:

\begin{equation}
E^{(1)} = E_{HK}[\rho _{0}^{(1)}] = \bra{\Psi ^{(1)}} H^{(1)} \ket{\Psi 
    ^{(1)}}
\end{equation}
\noindent Considerando $\rho ^{(2)}$ con su funci\'on de onda correspondiente
$\Psi ^{(2)}$,

\begin{equation}
E^{(1)} = \bra{\Psi ^{(1)}} H^{(1)} \ket{\Psi ^{(1)}} < \bra{\Psi ^{(2)}} 
H^{(1)} \ket{\Psi ^{(2)}}
\end{equation}

\noindent Por lo tanto la energ\'ia para la densidad electr\'onica del estado 
fundamental 
$\rho_{0}(r)$ es menor que la energ\'ia para otra densidad electr\'onica 
$\rho (r)$.
