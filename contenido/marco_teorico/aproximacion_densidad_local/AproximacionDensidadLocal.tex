En la aproximaci\'on de densidad local la energ\'ia de 
intercambio-correlaci\'on del 
sistema estudiado que posee una densidad electr\'onica no homog\'enea, es 
aproximada calculando la energ\'ia de intercambio-correlaci\'on de un sistema 
con densidad electr\'onica homog\'enea (gas de electrones homogeneo). Usando 
esta simplificaci\'on se puede expresar esta energ\'ia del siguiente modo.

\begin{equation}
E_{xc}^{LDA} [\rho ] = \int d^{3}r \rho (r) \varepsilon _{xc}^{unif} [\rho 
(r)] \textrm{ ,}
\end{equation}

\noindent donde el termino $\varepsilon _{xc}^{unif} [\rho (r)]$ es la 
energ\'ia de 
intercambio-correlaci\'on por electr\'on. Esta aproximaci\'on es valida para 
una densidad electr\'onica uniforme y para una densidad electr\'onica que varia 
lentamente cumpliendo la condici\'on \ref{variacion_lenta_densidad}.

\begin{equation} \label{variacion_lenta_densidad}
\frac{|\bigtriangledown \rho |}{\rho } \ll (3 \pi ^{2} \rho)^{1/3}
\end{equation}

\noindent La energ\'ia de intercambio-correlaci\'on por electr\'on $\varepsilon 
_{xc}^{unif} [\rho (r)]$ puede ser separada en un termino de intercambio y en 
un termino de correlaci\'on

\begin{equation}
   \varepsilon _{xc}^{unif} (\rho ) = \varepsilon _{x}^{unif} (\rho ) + 
   \varepsilon _{c}^{unif} (\rho )
\end{equation}

\noindent El termino de intercambio es conocido, ya que se puede expresar en 
forma 
anal\'itica para un gas de electrones homog\'eneo. 

\begin{equation}
   \varepsilon _{x}^{unif} (\rho ) = -\frac{3}{4} \left( \frac{3}{\pi } \right) 
   ^{1/3} \rho ^{1/3}
\end{equation}

\noindent El termino de correlaci\'on no puede ser expresado en forma 
anal\'itica, por lo 
que se utiliza el m\'etodo de montecarlo cu\'antico para varias densidades 
$\rho $ cuyos resultados son ajustado a una funci\'on parametrizada de $\rho $ 
que debe cumplir con dos condiciones expresadas en \ref{alta_baja_densidad}

\begin{equation} \label{alta_baja_densidad}
    \varepsilon _{c}^{unif} (\rho ) = \left\{ \begin{array}{ll}
    A \ln r_{s} B + C r_{s} \ln r_{s} & \textrm{; } r_{s} \to 0 \\ 
    \frac{D}{r_{s}} + \frac{E}{r_{s}^{3/2}} & \textrm{; } r_{s} \to \infty
    \end{array} \right. \textrm{ ,}
\end{equation}

\noindent donde \textbf{A,B,C,D,E} son constantes y $r_{s}$ es el radio de 
Wigner-Seitz, 
que es el radio m\'inimo que define un volumen donde se halla un solo 
electr\'on, y se encuentra relacionado con la densidad electr\'onica como se 
muestra en \ref{radio_wigner_seitz}

\begin{equation} \label{radio_wigner_seitz}
    r_{s} = \left( \frac{3}{4 \pi \rho } \right) ^{1/3}
\end{equation}