El m\'etodo asume que para un sistema de n-electrones, cada electr\'on no 
percibe a los otros como entidades independientes, sino como un campo promedio. 
Es decir que el sistema de n-electrones se convierte en un sistema 
mono-electr\'onico, donde cada electr\'on se mueve en una densidad promedio 
formada por el resto de electrones.

Con este modelo simplificado, se puede escribir la ecuaci\'on de onda del 
siguiente modo

%\begin{equation}
%\big [ - \sum _{i}^{n}\frac{1}{2} \nabla _{i}^{2} - \sum _{i} ^{N} 
%\frac{Z_{i}}{|R_{i}-r|} + \sum _{i \ne j} ^{n} \int \frac{|\phi 
%    _{i}(r^{\prime})|^{2}}{|r-r^{\prime }|}d^{3}r^{\prime} \big] \phi _{j} (r) 
%    = E_{j} \phi _{j} (r) \textrm{ ,}
%\end{equation}

\begin{equation}
\big [ - \frac{1}{2} \nabla _{i}^{2} - \sum _{i} ^{N} 
\frac{Z_{i}}{|R_{i}-r|} + \sum _{i \ne j} ^{n} \int \frac{|\phi 
    _{i}(r^{\prime})|^{2}}{|r-r^{\prime }|}d^{3}r^{\prime} \big] \phi _{j} (r) 
= E_{j} \phi _{j} (r) \textrm{ ,}
\end{equation}

\noindent donde el primer t\'ermino es la energ\'ia cin\'etica de los 
electrones. El 
segundo t\'ermino es el potencial externo $V_{ext}$ es decir la interacci\'on 
atractiva entre los electrones y los n\'ucleos. El tercer t\'ermino es el 
potencial de Hartree $V_{H}$ que proviene de la interacci\'on repulsiva de 
Coulomb entre cada electr\'on y el campo promedio generado por los dem\'as 
electrones.


\noindent Dado que los electrones son independientes, la energ\'ia total del 
sistema es 
la suma de las energ\'ias de cada mono-electr\'on.

\begin{equation}
    E = E_{1} + E_{2} + \dots + E_{n}
\end{equation}

\noindent Adem\'as, Hartree propuso que la funci\'on de onda del sistema, puede 
ser 
aproximada como el producto de las funciones de onda mono-electr\'onicas.

\begin{equation}
    \Psi = \phi_{1} \times \phi_{2} \times \dots \times \phi_{n}
\end{equation}

\noindent Tambi\'en debemos notar que la funci\'on de onda mono-electr\'onica 
que se 
busca se encuentra dentro del hamiltoniano en  la ecuaci\'on de onda, por lo 
que Hartree introdujo un m\'etodo para solucionar la ecuaci\'on de onda llamado 
m\'etodo de campo auto-consistente.

\noindent A pesar de que el modelo de Hartree tuvo \'exito al ser aplicado al 
hidr\'ogeno, no pudo hacer predicciones precisas para otros sistemas. Debido a 
las siguientes razones.

\begin{itemize}
    \item No sigue los dos principios b\'asicos de la mec\'anica cu\'antica: El 
    principio de antisimetr\'ia y el principio de exclusi\'on de Pauli.
    \item No toma en cuenta las energ\'ias de intercambio y correlaci\'on que 
    provienen de la naturaleza de los electrones.
\end{itemize}
