Luego que Hohenberg y Kohn enunciaran sus dos teoremas en 
1964, se asent\'o 
el uso de la densidad electr\'onica del estado fundamental, como una variable 
con la cual se pod\'ia obtener informaci\'on \'util del sistema estudiado. 
Pero, los teoremas no indicaban una forma de como usar la densidad 
electr\'onica de una manera practica. En 1965 Kohn y Sham desarrollaron un 
modelo de electrones independientes, que utilizaba la densidad electr\'onica 
como variable fundamental.

\noindent El hamiltoniano de un sistema de \textbf{n} electrones es:

\begin{equation}
    H = -\frac{1}{2} \sum _{i=1}^{n} \nabla _{i}^{2} - \sum 
    _{I=1}^{N} \sum _{i=1}^{n} \frac{Z_{I}}{|r_{i} - r_{I}|} + \frac{1}{2} 
    \sum _{i\neq j}^{n} \frac{1}{|r_{i} - r_{j}|}
\end{equation}

\noindent Seg\'un el m\'etodo de Hartree-Fock la energ\'ia de los \textbf{n} 
electrones interactuantes es:

\begin{equation}
    E = E_{k} + E_{ext} + E_{H} + E_{x} \textrm{ ,}
\end{equation}

\noindent donde $E_{k}$ es la energ\'ia cin\'etica, $E_{ext}$ es la energ\'ia 
del potencial externo, $E_{H}$ es la energ\'ia de Hartree y $E_{x}$ es la 
energ\'ia de intercambio.

\noindent En el enfoque Kohn-Sham, se asume que los electrones no interactuan y 
que el sistema se 
encuentra en su estado fundamental. De esta forma la 
energ\'ia cin\'etica de los electrones es.

\begin{equation}
    E_{k} = E_{k}^{no} + E_{k}^{in}
\end{equation}

\noindent donde $E_{k}^{no}$ es la energ\'ia cin\'etica no interactuante y  
$E_{k}^{in}$ es una correcci\'on a la energ\'ia cin\'etica no interactuante y 
puede considerarse como un termino de energ\'ia de correlaci\'on entre los 
electrones.  Adem\'as se agreg\'o un termino nuevo de energ\'ia de 
correlaci\'on 
que no se consider\'o en el m\'etodo de Hartree-Fock ($E_{c}^{in}$).

\noindent  Ahora reuniremos todos los t\'erminos de interacci\'on en un solo 
t\'ermino llamado energ\'ia de intercambio-correlaci\'on.

\begin{equation}
    E_{xc} = E_{x} + E_{c}^{in} + E_{k}^{in} = E_{x} + E_{c}
\end{equation}

\noindent Notamos que las energ\'ias $E_{c}^{in}$ y $E_{k}^{in}$ forman la 
energ\'ia de correlaci\'on $E_{c}$. A partir de esto, podemos escribir la 
energ\'ia del siguiente modo:

\begin{equation}
    E = E_{k}^{no} + E_{ext} + E_{H} + E_{xc}
\end{equation}

\noindent Es posible calcular los tres primeros terminos, pero el \'ultimo es 
desconocido y debe ser aproximado. El hamiltoniano correspondiente es

\begin{equation}\label {hamil_kohn_sham}
    H_{KS} = E_{k}^{no} + V_{ext} + V_{H} + V_{xc} = - \frac{\hbar ^{2}}{2m} 
    \nabla ^{2} + V_{ef}
\end{equation}

\noindent En \ref{hamil_kohn_sham} $V_{ef}$ es el potencial efectivo que 
contiene los tres t\'erminos potenciales, que definen un sistema de electrones 
no interactuantes, el cual es m\'as sencillo de calcular. Si se conociera con 
precisi\'on la forma del t\'ermino de intercambio-correlaci\'on, entonces la 
densidad electr\'onica y la energ\'ia del estado fundamental serian igual a la 
del sistema interactuante. 

\noindent Es necesario expresar los t\'erminos de la energ\'ia en funci\'on de 
la densidad electr\'onica para poder aplicar el principio variacional respecto 
de la densidad electr\'onica y hallar de esta forma las ecuaciones de Kohn-Sham.

\noindent La energ\'ia cin\'etica debe ser expresada como una suma de los 
orbitales de Kohn-Sham y no en funci\'on de la densidad electr\'onica debido a 
la derivada de segundo orden, pero los orbitales se encuentran directamente 
relacionados con la densidad electr\'onica por medio de $\rho = \sum _{i} |\phi 
_{i}|^{2}$, entonces la energ\'ia cin\'etica queda expresada en la forma de
\ref{cineticaKS}

\begin{equation} \label{cineticaKS}
    E_{k}^{no} = -\frac{1}{2} \sum _{i=1}^{n} \phi _{i}^{\ast}(r)\nabla_{i}^{2} 
    \phi_{i}(r)
\end{equation}

\noindent Para el caso de la energ\'ia externa, proveniente de la interacci\'on 
con los 
n\'ucleos, podemos expresarla en la forma de \ref{externaKS}

\begin{equation} \label{externaKS}
    E_{ext} = \int \phi ^{\ast }(r) V_{ext}(r) \phi (r) dr = \int V_{ext}(r) 
    \rho (r) dr
\end{equation}

\noindent En el caso de la energ\'ia de Hartree, podemos expresarla en la forma 
de 
\ref{hartreeKS}

\begin{equation} \label{hartreeKS}
    E_{H} = \int V_{H}(r) \rho (r) dr
\end{equation}

\noindent Finalmente en el caso de la energ\'ia de intercambio-correlaci\'on, 
se asumir\'a que se puede expresar en funci\'on de la densidad, ya que es 
necesario hacer aproximaciones que se revisaran con mayor detalle en secciones 
posteriores.

%             =
%            = =
%           =   =
%          =     =
%         =       = 
% ========         ==========
% ECUACIONES DE kOHN-SHAM
% ========         ===========
%         =       =
%          =     =
%           =   =
%            = =
%             =

\subsection{Ecuaciones de Kohn-Sham}

Para hallar las ecuaciones de Kohn-Sham se utiliza el m\'etodo de 
multiplicadores de lagrange, para lo cual definimos la restricci\'on expresada 
en \ref{restriccion_kohn_sham} dado que los orbitales deben ser ortonormales.


\begin{equation} \label{restriccion_kohn_sham}
    \int \phi _{i}^{\ast }(r) \phi _{j}(r)dr = \left \{ \begin{array}{ll}
        1 & i=j \\
        0 & i \neq j
        \end{array} \right. \textrm{ ,}
\end{equation}

\noindent a partir de \ref{restriccion_kohn_sham} obtenemos

\begin{equation}
     \int \phi _{i}^{\ast }(r) \phi _{j}(r)dr - 1 = 0
\end{equation}

\noindent Aplicando el m\'etodo de multiplicadores de Lagrange obtenemos

\begin{equation}
    \frac{\delta }{\delta \phi _{i}^{\ast }(r)} \left( E[\rho (r)] - \sum _{i} 
    \sum _{j}     \lambda _{ij} \left[ \int \phi _{i}^{\ast }(r) \phi _{j}(r) 
    dr - 1  \right] \right) = 0 \textrm{ ,}
\end{equation}

\noindent luego usando la regla de la cadena en 
la derivada funcional obtenemos

\begin{equation}
\frac{\delta E_{k}^{no}}{\delta \phi _{i}^{\ast }(r)} + \left[ \frac{\delta 
    E_{ext}}{\delta \rho (r)} + \frac{\delta E_{H}}{\delta \rho (r)} + 
\frac{\delta E_{ec}}{\delta \rho (r)} \right]\frac{\delta \rho (r)}{\delta 
    \phi _{i}^{\ast }(r)} - \sum _{j} \lambda_{ij} \phi _{j}(r) = 0 \textrm{ ,}
\end{equation}

\noindent donde $\lambda_{ij}$ es una matriz hermitiana que puede ser 
diagonalizada por una transformaci\'on unitaria de los orbitales.

\begin{equation}
    \left[ -\frac{1}{2} \nabla_{i}^{2} + V_{ext}  + V_{H} + V_{xc} - 
    \varepsilon_{i} \right] \phi_{i}(r) = 0 \textrm{ ,}
\end{equation}

\noindent reuniendo los t\'erminos $V_{ext}$, $V_{H}$ y $V_{xc}$ bajo el 
t\'ermino $V_{ef}$ que se conoce como potencial efectivo tenemos

\begin{equation}
    \left[ -\frac{1}{2}\nabla_{i}^{2} + V_{ef} - 
    \varepsilon_{i} \right] \phi_{i}(r) = 0
\end{equation}

\noindent Obtenemos la ecuaci\'on de Kohn-Sham

\begin{equation}
    \left[ -\frac{1}{2}\nabla_{i}^{2} + V_{ef} \right] \phi_{i}(r) = 
    \varepsilon_{i} \phi_{i}(r)
\end{equation}
